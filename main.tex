\documentclass[11pt,oneside,a4paper]{article}
\usepackage[margin=2cm]{geometry}

\usepackage[T1,T2A]{fontenc}
\usepackage[utf8]{inputenc}

\usepackage[english,bulgarian]{babel}
\usepackage{biblatex}

\usepackage{fancyhdr}
\usepackage{footmisc}
\usepackage{calc}

\usepackage{float}
\usepackage{caption}

\usepackage{amsmath,amsthm,amssymb}
\usepackage{mathtext}
\usepackage{mathtools}

\usepackage{array}

\usepackage{braket}

%    Q-circuit version 2
%    Copyright (C) 2004  Steve Flammia & Bryan Eastin
%    Last modified on: 9/16/2011
%
%    This program is free software; you can redistribute it and/or modify
%    it under the terms of the GNU General Public License as published by
%    the Free Software Foundation; either version 2 of the License, or
%    (at your option) any later version.
%
%    This program is distributed in the hope that it will be useful,
%    but WITHOUT ANY WARRANTY; without even the implied warranty of
%    MERCHANTABILITY or FITNESS FOR A PARTICULAR PURPOSE.  See the
%    GNU General Public License for more details.
%
%    You should have received a copy of the GNU General Public License
%    along with this program; if not, write to the Free Software
%    Foundation, Inc., 59 Temple Place, Suite 330, Boston, MA  02111-1307  USA

% Thanks to the Xy-pic guys, Kristoffer H Rose, Ross Moore, and Daniel Müllner,
% for their help in making Qcircuit work with Xy-pic version 3.8.  
% Thanks also to Dave Clader, Andrew Childs, Rafael Possignolo, Tyson Williams,
% Sergio Boixo, Cris Moore, Jonas Anderson, and Stephan Mertens for helping us test 
% and/or develop the new version.

\usepackage{xy}
\xyoption{matrix}
\xyoption{frame}
\xyoption{arrow}
\xyoption{arc}

\usepackage{ifpdf}
\ifpdf
\else
\PackageWarningNoLine{Qcircuit}{Qcircuit is loading in Postscript mode.  The Xy-pic options ps and dvips will be loaded.  If you wish to use other Postscript drivers for Xy-pic, you must modify the code in Qcircuit.tex}
%    The following options load the drivers most commonly required to
%    get proper Postscript output from Xy-pic.  Should these fail to work,
%    try replacing the following two lines with some of the other options
%    given in the Xy-pic reference manual.
\xyoption{ps}
\xyoption{dvips}
\fi

% The following resets Xy-pic matrix alignment to the pre-3.8 default, as
% required by Qcircuit.
\entrymodifiers={!C\entrybox}

\newcommand{\bra}[1]{{\left\langle{#1}\right\vert}}
\newcommand{\ket}[1]{{\left\vert{#1}\right\rangle}}
    % Defines Dirac notation. %7/5/07 added extra braces so that the commands will work in subscripts.
\newcommand{\qw}[1][-1]{\ar @{-} [0,#1]}
    % Defines a wire that connects horizontally.  By default it connects to the object on the left of the current object.
    % WARNING: Wire commands must appear after the gate in any given entry.
\newcommand{\qwx}[1][-1]{\ar @{-} [#1,0]}
    % Defines a wire that connects vertically.  By default it connects to the object above the current object.
    % WARNING: Wire commands must appear after the gate in any given entry.
\newcommand{\cw}[1][-1]{\ar @{=} [0,#1]}
    % Defines a classical wire that connects horizontally.  By default it connects to the object on the left of the current object.
    % WARNING: Wire commands must appear after the gate in any given entry.
\newcommand{\cwx}[1][-1]{\ar @{=} [#1,0]}
    % Defines a classical wire that connects vertically.  By default it connects to the object above the current object.
    % WARNING: Wire commands must appear after the gate in any given entry.
\newcommand{\gate}[1]{*+<.6em>{#1} \POS ="i","i"+UR;"i"+UL **\dir{-};"i"+DL **\dir{-};"i"+DR **\dir{-};"i"+UR **\dir{-},"i" \qw}
    % Boxes the argument, making a gate.
\newcommand{\meter}{*=<1.8em,1.4em>{\xy ="j","j"-<.778em,.322em>;{"j"+<.778em,-.322em> \ellipse ur,_{}},"j"-<0em,.4em>;p+<.5em,.9em> **\dir{-},"j"+<2.2em,2.2em>*{},"j"-<2.2em,2.2em>*{} \endxy} \POS ="i","i"+UR;"i"+UL **\dir{-};"i"+DL **\dir{-};"i"+DR **\dir{-};"i"+UR **\dir{-},"i" \qw}
    % Inserts a measurement meter.
    % In case you're wondering, the constants .778em and .322em specify
    % one quarter of a circle with radius 1.1em.
    % The points added at + and - <2.2em,2.2em> are there to strech the
    % canvas, ensuring that the size is unaffected by erratic spacing issues
    % with the arc.
\newcommand{\measure}[1]{*+[F-:<.9em>]{#1} \qw}
    % Inserts a measurement bubble with user defined text.
\newcommand{\measuretab}[1]{*{\xy*+<.6em>{#1}="e";"e"+UL;"e"+UR **\dir{-};"e"+DR **\dir{-};"e"+DL **\dir{-};"e"+LC-<.5em,0em> **\dir{-};"e"+UL **\dir{-} \endxy} \qw}
    % Inserts a measurement tab with user defined text.
\newcommand{\measureD}[1]{*{\xy*+=<0em,.1em>{#1}="e";"e"+UR+<0em,.25em>;"e"+UL+<-.5em,.25em> **\dir{-};"e"+DL+<-.5em,-.25em> **\dir{-};"e"+DR+<0em,-.25em> **\dir{-};{"e"+UR+<0em,.25em>\ellipse^{}};"e"+C:,+(0,1)*{} \endxy} \qw}
    % Inserts a D-shaped measurement gate with user defined text.
\newcommand{\multimeasure}[2]{*+<1em,.9em>{\hphantom{#2}} \qw \POS[0,0].[#1,0];p !C *{#2},p \drop\frm<.9em>{-}}
    % Draws a multiple qubit measurement bubble starting at the current position and spanning #1 additional gates below.
    % #2 gives the label for the gate.
    % You must use an argument of the same width as #2 in \ghost for the wires to connect properly on the lower lines.
\newcommand{\multimeasureD}[2]{*+<1em,.9em>{\hphantom{#2}} \POS [0,0]="i",[0,0].[#1,0]="e",!C *{#2},"e"+UR-<.8em,0em>;"e"+UL **\dir{-};"e"+DL **\dir{-};"e"+DR+<-.8em,0em> **\dir{-};{"e"+DR+<0em,.8em>\ellipse^{}};"e"+UR+<0em,-.8em> **\dir{-};{"e"+UR-<.8em,0em>\ellipse^{}},"i" \qw}
    % Draws a multiple qubit D-shaped measurement gate starting at the current position and spanning #1 additional gates below.
    % #2 gives the label for the gate.
    % You must use an argument of the same width as #2 in \ghost for the wires to connect properly on the lower lines.
\newcommand{\control}{*!<0em,.025em>-=-<.2em>{\bullet}}
    % Inserts an unconnected control.
\newcommand{\controlo}{*+<.01em>{\xy -<.095em>*\xycircle<.19em>{} \endxy}}
    % Inserts a unconnected control-on-0.
\newcommand{\ctrl}[1]{\control \qwx[#1] \qw}
    % Inserts a control and connects it to the object #1 wires below.
\newcommand{\ctrlo}[1]{\controlo \qwx[#1] \qw}
    % Inserts a control-on-0 and connects it to the object #1 wires below.
\newcommand{\targ}{*+<.02em,.02em>{\xy ="i","i"-<.39em,0em>;"i"+<.39em,0em> **\dir{-}, "i"-<0em,.39em>;"i"+<0em,.39em> **\dir{-},"i"*\xycircle<.4em>{} \endxy} \qw}
    % Inserts a CNOT target.
\newcommand{\qswap}{*=<0em>{\times} \qw}
    % Inserts half a swap gate.
    % Must be connected to the other swap with \qwx.
\newcommand{\multigate}[2]{*+<1em,.9em>{\hphantom{#2}} \POS [0,0]="i",[0,0].[#1,0]="e",!C *{#2},"e"+UR;"e"+UL **\dir{-};"e"+DL **\dir{-};"e"+DR **\dir{-};"e"+UR **\dir{-},"i" \qw}
    % Draws a multiple qubit gate starting at the current position and spanning #1 additional gates below.
    % #2 gives the label for the gate.
    % You must use an argument of the same width as #2 in \ghost for the wires to connect properly on the lower lines.
\newcommand{\ghost}[1]{*+<1em,.9em>{\hphantom{#1}} \qw}
    % Leaves space for \multigate on wires other than the one on which \multigate appears.  Without this command wires will cross your gate.
    % #1 should match the second argument in the corresponding \multigate.
\newcommand{\push}[1]{*{#1}}
    % Inserts #1, overriding the default that causes entries to have zero size.  This command takes the place of a gate.
    % Like a gate, it must precede any wire commands.
    % \push is useful for forcing columns apart.
    % NOTE: It might be useful to know that a gate is about 1.3 times the height of its contents.  I.e. \gate{M} is 1.3em tall.
    % WARNING: \push must appear before any wire commands and may not appear in an entry with a gate or label.
\newcommand{\gategroup}[6]{\POS"#1,#2"."#3,#2"."#1,#4"."#3,#4"!C*+<#5>\frm{#6}}
    % Constructs a box or bracket enclosing the square block spanning rows #1-#3 and columns=#2-#4.
    % The block is given a margin #5/2, so #5 should be a valid length.
    % #6 can take the following arguments -- or . or _\} or ^\} or \{ or \} or _) or ^) or ( or ) where the first two options yield dashed and
    % dotted boxes respectively, and the last eight options yield bottom, top, left, and right braces of the curly or normal variety.  See the Xy-pic reference manual for more options.
    % \gategroup can appear at the end of any gate entry, but it's good form to pick either the last entry or one of the corner gates.
    % BUG: \gategroup uses the four corner gates to determine the size of the bounding box.  Other gates may stick out of that box.  See \prop.

\newcommand{\rstick}[1]{*!L!<-.5em,0em>=<0em>{#1}}
    % Centers the left side of #1 in the cell.  Intended for lining up wire labels.  Note that non-gates have default size zero.
\newcommand{\lstick}[1]{*!R!<.5em,0em>=<0em>{#1}}
    % Centers the right side of #1 in the cell.  Intended for lining up wire labels.  Note that non-gates have default size zero.
\newcommand{\ustick}[1]{*!D!<0em,-.5em>=<0em>{#1}}
    % Centers the bottom of #1 in the cell.  Intended for lining up wire labels.  Note that non-gates have default size zero.
\newcommand{\dstick}[1]{*!U!<0em,.5em>=<0em>{#1}}
    % Centers the top of #1 in the cell.  Intended for lining up wire labels.  Note that non-gates have default size zero.
\newcommand{\Qcircuit}{\xymatrix @*=<0em>}
    % Defines \Qcircuit as an \xymatrix with entries of default size 0em.
\newcommand{\link}[2]{\ar @{-} [#1,#2]}
    % Draws a wire or connecting line to the element #1 rows down and #2 columns forward.
\newcommand{\pureghost}[1]{*+<1em,.9em>{\hphantom{#1}}}
    % Same as \ghost except it omits the wire leading to the left. 


\newcommand{\figref}[1]{\figurename~\ref{#1}}

\usepackage[unicode]{hyperref}
\hypersetup{
    colorlinks=true,
    linkcolor=blue,
    filecolor=magenta,      
    urlcolor=cyan,
}

\setlength\headheight{3.5em}
\setlength\footskip{14pt}

\addtolength{\footnotesep}{0.8em} \renewcommand{\thefootnote}{\textbf{\arabic{footnote}}}

\addbibresource{cite.bib}

\renewcommand*{\thesection}{\arabic{section}}

\pagestyle{fancy}
\fancyhf{}
\rhead{Алгоритъм за засичане и коригиране на грешки в квантови системи}
\lhead{Цветелин Цецков}
\rfoot{Page \thepage}

\title{Тема по квантива информатика: \\ Алгоритъм за засичане и коригиране на грешки в квантови системи\\ Quantum Error Correction}
\author{Цветелин Цецков ФН:82130 }
\date{\today}

\begin{document}

\maketitle
\tableofcontents
\newpage

\begin{abstract}
При квантовите операции и пренос на състояния винаги е възможно да настъпи т.н. decoherence - от чисти състояния, представени с вектори, до нечисти състояния, представени от матрици на плътността. Или да останем в чисто състояние, но околната среда да "измери"\ състоянието вместо нас. Тогава спрямо Проекционният постулат системата ще премине със скок в ново състояние спрямо изчислителния базис.
\end{abstract}

\section{Класически подход}
Можем да заемем подход от един от първите подходи за корекции на грешки в класическите компютри - да повторим бит няколко пъти\footnote{
Класическите \href{https://en.wikipedia.org/wiki/Repetition_code}{Кодировки с повторение}. Основен принци, на които е повторението на един и същи бит разчитайки на факта, че повече на брой грешки са по-малко вероятни.
}. Това не е проблем, ако нашето състояние е едно от базисните 
$\ket{0}, \ket{1}, \ket{+}$ или $\ket{-}$.\\
Ако кубитът, който кодираме е $\ket{\psi}\in\set{\ket{0}, \ket{1}, \ket{+}, \ket{-}}$. To нашата кодировка с повторение би изглеждала по следния начин:
\begin{equation} \label{eq:copy_simple}
    \ket{\phi} = \ket{\psi} \otimes \ket{\psi} \otimes \ket{\psi}
\end{equation}
Тази кодировка се основава на факта, че ако $p$ е вероятността за обръщане на един бит при преноста(или изчислението). То вероятността за:\\
1 грешка е от порядъка на $p.(1-p).(1-p) \rightarrow p$ \\
2 грешки е от порядъка на $p.p.(1-p) \rightarrow p^2$ \\
3 грешки е от порядъка на $p.p.p \rightarrow p^3$

\section{Преход към квантова система}
Но обаче ако имаме произволно състояние $\ket{\psi} = \alpha\ket{0} + \beta\ket{1}$, то за него не бихме могли да приложим \eqref{eq:copy_simple}\footnote{
Защото е в сила \href{https://en.wikipedia.org/wiki/No-cloning_theorem}{Теоремата}, че не можем да клонираме произволно състояние 
}\\
Трябва да се измисли начин, по който да можем да поправяме грешки в кодировката без да се налага да измерваме системата, или поне не директно, защото тогава ще се приложи Проекционният постулат и ще загубим състоянието.\\
Вместо да правим:
\begin{equation} \label{eq:copy_simple_not_working}
    \ket{\phi} = \ket{\psi} \otimes \ket{\psi} \otimes \ket{\psi}
\end{equation}

\subsection{Кодировки на един кубит}
Една от най-простите кодировки би била да използваме помощни битове, за да запазим състоянието.
\begin{figure}[H]
    \centering
    \begin{math}
    \begin{array}{c}
    \Qcircuit @C=1em @R=.7em {
    \lstick{\ket{\psi}} 
        & \ctrl{1} 
        & \ctrl{2} 
        & \qw 
        & \rstick{\ket{\psi}}\\
    \lstick{\ket{0}} 
        & \targ 
        & \qw
        & \qw 
        & \rstick{\ket{ans_1}}\\
    \lstick{\ket{0}} 
        & \qw 
        & \targ
        & \qw
        & \rstick{\ket{ans_2}}\\
    }
\end{array}
\end{math}
    \caption{Примерно кодиране на неизвестно състояние $\ket{\psi}$}
    \label{fig:sample_coding}
\end{figure}
Ако $\ket{\psi} = \alpha\ket{0} + \beta\ket{1}$, то след кодировката получаваме:
\begin{equation}
    \ket{\phi} = \alpha\ket{000} + \beta\ket{111} \neq \ket{\psi} \otimes \ket{\psi} \otimes \ket{\psi}
\end{equation}
Нека за пълнота приемем $\ket{\psi} = \alpha\ket{0} + \beta\ket{1}$ и изпишем матрицата на кодировката.
\paragraph{Нека преди това да припомним матриците на трансформациите}
Както и операциите, които прилагаме към трансформациите, за да можем да ги приложим към нашия "квантов регистър"\footnote{
В аналог с класическите компютри този термин ще бъде използван за съставна система от 2 или повече кубита
}
\begin{equation}
    CNOT =\begin{bmatrix}
    1 & 0 & 0 & 0\\
    0 & 1 & 0 & 0\\
    0 & 0 & 0 & 1\\
    0 & 0 & 1 & 0
    \end{bmatrix}
    ,\;
    SWAP = \begin{bmatrix}
    1 & 0 & 0 & 0\\
    0 & 0 & 1 & 0\\
    0 & 1 & 0 & 0\\
    0 & 0 & 0 & 1
    \end{bmatrix}
    \label{eq:cnot_and_swap_matrices}
\end{equation}
Матриците на използваните операции в \figref{fig:sample_coding}(на нея се вижда само $CNOT$ операцията, но когато изпишем алгоритъма в явен вид ще стане ясно, че се ползва и $SWAP$ операцията)

\begin{figure}[H]
    \centering
    \begin{math}
    \begin{array}{c}
    \Qcircuit @C=1em @R=.7em {
    & \gate{U} & \gate{W} & \qw \\
    }
\end{array}
\end{math}
    \caption{Последнователно прилагане на операции}
    \label{fig:sequential_gates_rule}
\end{figure}
Тогава тази операция може да се замести с една единствена получена по правилото: $R = W.U$
\begin{figure}[H]
    \centering
    \begin{math}
    \begin{array}{c}
    \Qcircuit @C=1em @R=.7em {
    & \gate{R = W.U} & \qw \\
    }
\end{array}
\end{math}
    \caption{Последнователно прилагане на операции}
    \label{fig:sequential_gates_rule_res}
\end{figure}
Когато обаче се случи да имаме операции, които действат само върху част от квантовия регистър:
\begin{figure}[H]
    \centering
    \begin{math}
    \begin{array}{c}
    \Qcircuit @C=1em @R=.7em {
    & \gate{U} & \qw \\
    & \gate{W} & \qw \\
    }
\end{array}
\end{math}
    \caption{Паралелно прилагане на операции}
    \label{fig:parallel_gates}
\end{figure}
Тогава можем да заместим двете с трансформация която действа едновременно и на двата кубита по правилото: $R = U \otimes\footnote{
С този символ означаваме тензорно произведение между 2 матрици. Още наричано  \href{https://en.wikipedia.org/wiki/Kronecker_product}{произведение на Кронекер}
} W$
\begin{figure}[H]
    \centering
    \begin{math}
    \begin{array}{c}
    \Qcircuit @C=1em @R=.7em {
    & \multigate{1}{R = U \otimes W} & \qw \\
    & \ghost{R = U \otimes W} & \qw \\
    }
\end{array}
\end{math}
    \caption{Паралелно прилагане на операции}
    \label{fig:parallel_gates_res}
\end{figure}
\paragraph{Явен вид на кодировката}
Сега е необходимо да изпишем по-подробно предишната схема от \figref{fig:sample_coding} във \figref{fig:sample_coding_expanded}.
\begin{figure}[H]
    \centering
    \begin{math}
    \begin{array}{c}
    \Qcircuit @C=1em @R=.7em {
    \lstick{\ket{\psi}} 
        & \ctrl{1} 
        & \qswap
        & \gate{I \footnotemark}
        & \qswap
        & \qw 
        & \rstick{\ket{\psi}}\\
    \lstick{\ket{0}} 
        & \targ 
        & \qswap \qwx
        & \ctrl{1} 
        & \qswap \qwx
        & \qw 
        & \rstick{\ket{ans_1}}\\
    \lstick{\ket{0}} 
        & \gate{I}
        & \gate{I}
        & \targ
        & \gate{I}
        & \qw
        & \rstick{\ket{ans_2}}\\
    }
\end{array}
\end{math}
    \caption{Примерно кодиране на неизвестно състояние $\ket{\psi}$}
    \label{fig:sample_coding_expanded}
\end{figure}
\footnotetext{
Използваме $I$ като операцията идентитет - не осъществява никаква трансформация и има матрица $\big[\begin{smallmatrix}1&0\\0&1\end{smallmatrix}\big]$
}
Това ще ни помогне да видим кои операции ще са ни необходими(ако построяваме и физически) и какви операции трябва да приложим между матриците на трансформациите. В случая множеството $\set{I, CNOT, SWAP}$ е напълно достатъчно, за да построим нашият алгоритъм.\\
Така можем да изведем формула за нашата кодировка\footnote{
Избраното име на трансформацията идва от анг. "encode" - кодирам, от там и кодировка. Прието е да се изписват имената на трансформациите на латиница с латински букви, за да има консистентност поне между чертежи, докато текстовете са различни.
}:
\begin{equation}
    ENCODE_1 = 
    \underbrace{(SWAP \otimes I)}_\text{Четвърта колона}
    .
    \underbrace{(I \otimes CNOT)}_\text{Трета колона}
    .
    \underbrace{(SWAP \otimes I)}_\text{Втора колона}
    .
    \underbrace{(CNOT \otimes I)}_\text{Първа колона}
\label{eq:encode_matrix_formula}
\end{equation}

\eqref{eq:encode_matrix_formula} представлява формула за изчисление на матрицата на нашата кодировка(за един кубит). Замествайки стойностите от \eqref{eq:cnot_and_swap_matrices} в нея получаваме "кодировъчната матрица":
\begin{equation} 
    ENCODE_1 = \begin{bmatrix}
    1 & 0 & 0 & 0 & 0 & 0 & 0 & 0\\
    0 & 1 & 0 & 0 & 0 & 0 & 0 & 0\\
    0 & 0 & 1 & 0 & 0 & 0 & 0 & 0\\
    0 & 0 & 0 & 1 & 0 & 0 & 0 & 0\\
    0 & 0 & 0 & 0 & 0 & 0 & 0 & 1\\
    0 & 0 & 0 & 0 & 0 & 0 & 1 & 0\\
    0 & 0 & 0 & 0 & 0 & 1 & 0 & 0\\
    0 & 0 & 0 & 0 & 1 & 0 & 0 & 0
    \end{bmatrix}
    \label{eq:encode_matrix_res}
\end{equation}
От сега нататък $ENCODE_1$ ще бъде използвана като кодировка на 3 кубита по метода обяснен по-горе.\\
Интересното на тази кодировка е, че посредством тези $CNOT$ операции, цялата система става "entangled"(преплетена). Именно от този феномен идва всеизветната фраза на Айнщайн - "spooky action at a distance", тъй като ако измерим един кубит на системата веднага ще знаем стойността на другите два. \cite{AE-MB-letters}
\paragraph{Примерни изчисления с кодировъчната матрица}
Ще извършим кодировка на всеки от базисните вектори на състоянията и в правоъгълния базис\footnote{
Прието е правоъгълният базис да се състои от $\set{\ket{1}, \ket{0}}$
}, и диагоналния\footnote{
Диагоналния базис се състои от $\set{\ket{+}, \ket{-}}$, който може да се получи от правоъгълния по следния начин $\ket{\pm} = \frac{1}{\sqrt{2}}\ket{0} \pm \frac{1}{\sqrt{2}}\ket{1}$
}
\begin{equation}
    \ket{0} = \begin{pmatrix}1\\0\end{pmatrix}
\end{equation}
НЕ можем просто така да приложим $ENCODE_1$ върху $\ket{0}$, защотое само един кубит, докато $ENCODE_1$ действа върху 3. Затова конструираме състояние $\ket{000} = \ket{0} \otimes \ket{0} \otimes \ket{0}$
\begin{equation}
    \ket{000} = \ket{0} \otimes \ket{0} \otimes \ket{0} = \begin{pmatrix}1\\0\end{pmatrix} \otimes \begin{pmatrix}1\\0\end{pmatrix} \otimes \begin{pmatrix}1\\0\end{pmatrix} = 
    \begin{pmatrix}1\\0\\0\\0\\0\\0\\0\\0\end{pmatrix}
\end{equation}
\begin{equation}
    ENCODE_1 . \ket{000} = ENCODE_1 .
    \begin{pmatrix}1\\0\\0\\0\\0\\0\\0\\0\end{pmatrix} = 
    \begin{bmatrix}
    1 & 0 & 0 & 0 & 0 & 0 & 0 & 0\\
    0 & 1 & 0 & 0 & 0 & 0 & 0 & 0\\
    0 & 0 & 1 & 0 & 0 & 0 & 0 & 0\\
    0 & 0 & 0 & 1 & 0 & 0 & 0 & 0\\
    0 & 0 & 0 & 0 & 0 & 0 & 0 & 1\\
    0 & 0 & 0 & 0 & 0 & 0 & 1 & 0\\
    0 & 0 & 0 & 0 & 0 & 1 & 0 & 0\\
    0 & 0 & 0 & 0 & 1 & 0 & 0 & 0
    \end{bmatrix} . \begin{pmatrix}1\\0\\0\\0\\0\\0\\0\\0\end{pmatrix}
    = \begin{pmatrix}1\\0\\0\\0\\0\\0\\0\\0\end{pmatrix} = \ket{000}
    \label{eq:simple_encode_000}
\end{equation}
Разглеждаме случая за $\ket{1}$
\begin{equation}
    \ket{1} = \begin{pmatrix}0\\1\end{pmatrix}
\end{equation}
Аналогично не можем да приложим кодировъчната матрица, затова конструираме състоянието $\ket{4}\footnote{
С цел по-кратък запис можем да използваме десетичния запис на квантовия регистър когато това е възможно, напр $\ket{101} = \ket{5}$
} = \ket{100} = \ket{1} \otimes \ket{0} \otimes \ket{0}$
\begin{equation}
    \ket{4} = \ket{100} = \ket{1} \otimes \ket{0} \otimes \ket{0} = \begin{pmatrix}0\\1\end{pmatrix} \otimes \begin{pmatrix}1\\0\end{pmatrix} \otimes \begin{pmatrix}1\\0\end{pmatrix} = \begin{pmatrix}0\\0\\0\\0\\1\\0\\0\\0\end{pmatrix}
\end{equation}
\begin{equation}
    ENCODE_1 . \ket{100} = ENCODE_1 .
    \begin{pmatrix}0\\0\\0\\0\\1\\0\\0\\0\end{pmatrix} = 
    \begin{bmatrix}
    1 & 0 & 0 & 0 & 0 & 0 & 0 & 0\\
    0 & 1 & 0 & 0 & 0 & 0 & 0 & 0\\
    0 & 0 & 1 & 0 & 0 & 0 & 0 & 0\\
    0 & 0 & 0 & 1 & 0 & 0 & 0 & 0\\
    0 & 0 & 0 & 0 & 0 & 0 & 0 & 1\\
    0 & 0 & 0 & 0 & 0 & 0 & 1 & 0\\
    0 & 0 & 0 & 0 & 0 & 1 & 0 & 0\\
    0 & 0 & 0 & 0 & 1 & 0 & 0 & 0
    \end{bmatrix} . \begin{pmatrix}0\\0\\0\\0\\1\\0\\0\\0\end{pmatrix}
    = \begin{pmatrix}0\\0\\0\\0\\0\\0\\0\\1\end{pmatrix} = \ket{111} = \ket{7}
    \label{eq:simple_encode_100}
\end{equation}
Сега за диагоналния базис:
\begin{equation}
    \ket{+} = \frac{\ket{0} + \ket{1}}{\sqrt{2}} = \frac{1}{\sqrt{2}}\begin{pmatrix}1\\1\end{pmatrix}
\end{equation}
\begin{equation}
    \ket{+00} = \ket{+} \otimes \ket{0} \otimes \ket{0} = 
    \frac{1}{\sqrt{2}}\begin{pmatrix}1\\1\end{pmatrix}
    \otimes \begin{pmatrix}1\\0\end{pmatrix}
    \otimes \begin{pmatrix}1\\0\end{pmatrix}
    = \frac{1}{\sqrt{2}} \begin{pmatrix}1\\1\\0\\0\\0\\0\\0\\0\end{pmatrix} 
    = \frac{1}{\sqrt{2}}\ket{000} + \frac{1}{\sqrt{2}}\ket{001}
\end{equation}
\begin{equation}
\begin{multlined}
    ENCODE_1 . \ket{+00} = ENCODE_1 .
    \frac{1}{\sqrt{2}}\begin{pmatrix}1\\1\\0\\0\\0\\0\\0\\0\end{pmatrix} = \\
    = \begin{bmatrix}
    1 & 0 & 0 & 0 & 0 & 0 & 0 & 0\\
    0 & 1 & 0 & 0 & 0 & 0 & 0 & 0\\
    0 & 0 & 1 & 0 & 0 & 0 & 0 & 0\\
    0 & 0 & 0 & 1 & 0 & 0 & 0 & 0\\
    0 & 0 & 0 & 0 & 0 & 0 & 0 & 1\\
    0 & 0 & 0 & 0 & 0 & 0 & 1 & 0\\
    0 & 0 & 0 & 0 & 0 & 1 & 0 & 0\\
    0 & 0 & 0 & 0 & 1 & 0 & 0 & 0
    \end{bmatrix} . \frac{1}{\sqrt{2}}\begin{pmatrix}1\\1\\0\\0\\0\\0\\0\\0\end{pmatrix} = \frac{1}{\sqrt{2}}\begin{pmatrix}1\\1\\0\\0\\0\\0\\0\\0\end{pmatrix} = \ket{+00}
\end{multlined}
\label{eq:simple_encode_+00}
\end{equation}

И последно случая:
\begin{equation}
    \ket{-} = \frac{\ket{0} - \ket{1}}{\sqrt{2}} = \frac{1}{\sqrt{2}}\begin{pmatrix}1\\-1\end{pmatrix}
\end{equation}
\begin{equation}
    \ket{-00} = \ket{-} \otimes \ket{0} \otimes \ket{0}
    = \frac{1}{\sqrt{2}}\begin{pmatrix}1\\-1\end{pmatrix}
    \otimes \begin{pmatrix}1\\0\end{pmatrix}
    \otimes \begin{pmatrix}1\\0\end{pmatrix} =
    \frac{1}{\sqrt{2}} \begin{pmatrix}1\\-1\\0\\0\\0\\0\\0\\0\end{pmatrix}
\end{equation}
\begin{equation}
\begin{multlined}
    ENCODE_1 . \ket{-00} = ENCODE_1 .
    \frac{1}{\sqrt{2}}\begin{pmatrix}1\\-1\\0\\0\\0\\0\\0\\0\end{pmatrix} = \\
    = \begin{bmatrix}
    1 & 0 & 0 & 0 & 0 & 0 & 0 & 0\\
    0 & 1 & 0 & 0 & 0 & 0 & 0 & 0\\
    0 & 0 & 1 & 0 & 0 & 0 & 0 & 0\\
    0 & 0 & 0 & 1 & 0 & 0 & 0 & 0\\
    0 & 0 & 0 & 0 & 0 & 0 & 0 & 1\\
    0 & 0 & 0 & 0 & 0 & 0 & 1 & 0\\
    0 & 0 & 0 & 0 & 0 & 1 & 0 & 0\\
    0 & 0 & 0 & 0 & 1 & 0 & 0 & 0
    \end{bmatrix} . \frac{1}{\sqrt{2}}\begin{pmatrix}1\\-1\\0\\0\\0\\0\\0\\0\end{pmatrix} = \frac{1}{\sqrt{2}}\begin{pmatrix}1\\-1\\0\\0\\0\\0\\0\\0\end{pmatrix} = \ket{-00}
\end{multlined}
\label{eq:simple_encode_-00}
\end{equation}

Интересно нещо, което можем да забележим е, че диагоналните базиси и $\ket{0}$ остават "същите", докато само $\ket{1}$ променя стойността си(вж. \eqref{eq:simple_encode_100})

\subsection{За шума}
Той може да бъде под най-различна форма и да "разстрои" нашето състояние по много начини. Може да бъде породен от множество фактори. Околната среда да "измери" нашето състояние и чрез проекционния посулат да загубим подготвеното състояние. Или пък от чисто състояние да го превърне в матрица на плътността, с която не е удобно да се работи за изчисления. Дори физическите трансформации, които прилагаме на кубита може да разстроят неговото състояние и да загубим нашите данни.\\\
А относно каква форма могат да приемат тези изменения, те могат да бъдат
\begin{enumerate}
    \item \label{item:bitflip} Обръщане на бит\footnote{т.нар. bit-flip от класическите системи, където например дадена система $\alpha_1\alpha_2\alpha_3$ преминава в системата: $\alpha_1\overline{\alpha_2}\alpha_3$}
    \item\label{item:phaseflip} Обръщане на фазa\footnote{
    т. нар. phase-flip, когато състоянието $\alpha\ket{0} + \beta\ket{0}$ преминава в $\alpha\ket{0} - \beta\ket{0}$.
    \label{ftnt:phase_flip}
    }
    \item\label{item:both} Линейна комбинация комбинация на \ref{item:bitflip} и \ref{item:phaseflip}
\end{enumerate}
В известен смисъл построения $ENCODE_1$ ни защитава само от \ref{item:bitflip}. Обаче ако се загледаме хубаво в обръщането на фазата можем да забележим, че обръщането на фазата в правоъгълния базис съответства на обръщане на бита в диагоналния, така че можем да продължим да изграждаме алгоритъма за корекция на грешки в правоъгълния базис и в последствие да го допълним и до диагонален базис, за да коригираме грешка и в него.\\\
В крайната диаграма ще използваме следното означение за шума от околната среда и грешката по време на опeрациите
\begin{figure}[H]
    \centering
    \begin{math}
    \begin{array}{c}
    \Qcircuit @C=1em @R=.7em {
        & \qw
        & \multigate{2}{Noise} 
        & \qw \\
    \lstick{\vdots} 
        & \qw 
        & \ghost{Noise} 
        & \qw 
        & \rstick{\vdots} \\
    & \qw 
        & \ghost{Noise}
        & \qw\\
    }
\end{array}
\end{math}
    \caption{Означението, че системата преминава през шум}
    \label{fig:noise_gate}
\end{figure}

\subsection{Декодиране на състоянието}
\paragraph{В общия случай} получаваме неизвестно състояние
\begin{equation}
    \ket{\psi} = \alpha_1\ket{0} + \alpha_2\ket{1} + \alpha_3\ket{2} + \alpha_4\ket{3} + \alpha_5\ket{4} + \alpha_6\ket{5} + \alpha_6\ket{6} + \alpha_7\ket{7}
\end{equation}
След като знаем, че идващото състояние е получено чрез нашата кодировка, можем да сме сигурни, че е разложимо(или поне трябва да бъде). Затова можем да си го мислим като получено от даден кубит $\ket{\phi} = \alpha\ket{0} + \beta\ket{1}$\\
Но не можем просто да започнем да измерваме, тъй като ще загубим информацията заради проекционния постулат. Затова е необходимо отново да включим т.нар. ancilla bits\footnote{
Бит, който не допринася за изчислението, а е необходим само за засичането и поправянето на грешки. \href{https://en.wikipedia.org/wiki/Ancilla_bit}{Ancilla bit}
}. Ще ги използваме да засечем грешките, които са възникнали и ще измерваме само тях, без да докосваме състоянието.

\begin{figure}[H]
    \centering
    \begin{math}
    \begin{array}{c}
    \Qcircuit @C=1em @R=1em {
    \lstick{}
        & \ctrl{3}
        & \qw 
        & \ctrl{4}
        & \qw 
        & \qw \\
    \lstick{\ket{\psi}} 
        & \qw 
        & \ctrl{2}
        & \qw 
        & \qw
        & \qw \\
    \lstick{}
        & \qw 
        & \qw 
        & \qw
        & \ctrl{2}
        & \qw \\
    \lstick{\ket{0}}
        & \targ
        & \targ
        & \qw
        & \qw 
        & \qw 
        & \rstick{\ket{x_1}}\\
    \lstick{\ket{0}}
        & \qw
        & \qw
        & \targ
        & \targ
        & \qw 
        & \rstick{\ket{x_2}}\\
    }
\end{array}
\end{math}
    \caption{Декодиране на полученото съсояние}
    \label{fig:sample_decode}
\end{figure}
Сега можем да разсъждаваме за $\ket{x_1}$ и  $\ket{x_2}$\footnote{
Тези разсъждения са взети от лекциите с цел пълнота на представянето.
}.\\
\begin{math}
\ket{x_1,x_2} = \begin{cases}
\ket{00} \rightarrow \text{Не правим нищо.} \\
\ket{01} \rightarrow \text{Прилагаме Pauli-X\footnotemark на третия кубит.}\\
\ket{10} \rightarrow \text{Прилагаме Pauli-X на втория кубит.}\\
\ket{11} \rightarrow \text{Прилагаме Pauli-X на първия кубит.}\\
\end{cases}
\end{math}
\footnotetext{
\href{https://en.wikipedia.org/wiki/Quantum_logic_gate\#Pauli-X_gate}{Pauli-X} трансформацията, кръстена на Волфганг Паули. Представлява класическата $NOT$ операция и има матрица $\left[\begin{smallmatrix}0&1\\1&0\end{smallmatrix}\right]$
}
И това би проработило и би ни дало верен резултат. Нека разпишем случаите и да се убедим:
\begin{enumerate}
    \item $\ket{x_1,x_2} = \ket{00}: \ket{x_1} = 0 \rightarrow \ket{\psi_1} = \ket{\psi_2}$ и $\ket{x_2} = 0 \rightarrow \ket{\psi_2} = \ket{\psi_3}$\footnote{
    Използваме $\ket{\psi_i}$ за означение на $i$-тия кубит в системата
    }
    \item $\ket{x_1,x_2} = \ket{01}: \ket{x_1} = 0 \rightarrow \ket{\psi_1} = \ket{\psi_2}$, но $\ket{x_2} = 1 \rightarrow \ket{\psi_1} \neq \ket{\psi_3} \rightarrow$ третия се различава от останалите
    \item $\ket{x_1,x_2} = \ket{01}: \ket{x_1} = 1 \rightarrow \ket{\psi_1} \neq \ket{\psi_2}$ и $\ket{x_2} = 0 \rightarrow \ket{\psi_1} = \ket{\psi_3} \rightarrow$ втория се различава от останалите
    \item $\ket{x_1,x_2} = \ket{11}: \ket{x_1} = 1 \rightarrow \ket{\psi_1} \neq \ket{\psi_2}$ и $\ket{x_2} = 0 \rightarrow \ket{\psi_1} \neq \ket{\psi_3} \rightarrow$ първия се различава от останалите
\end{enumerate}
Ако измерим кое да е от гореспоменатите спрямо проекционния постулат $\ket{\psi}$ ще премине със скок в съответното състояние, което преполагаемо имаме.\\
Например, ако измерим $\ket{00}$ ще имаме информация, че всички битове са еднакви следователно ще имаме $\braket{000 + 111|\psi}$, аналогично ако $\ket{x_1,x_2} = \ket{01}$ то ще знаем\footnote{
Под "знаем" се има предвид, че можем да разсъждаваме за нещо като за дадено.
}, че първия и третия кубит на системата се различават и можем да проектираме $\braket{110+001|\psi}$ и вече се вижда всъщност от къде идва решението за прилагане на $Pauli-X$ към третия \cite{dirac}. Аналогично и с останалите случаи: проектираме върху линейната комбинация, за която сме сигурни, че представлява състоянието на квантовия регистър.

\paragraph{Малка хитрина}
Mожем да спестим няколко операции като всъщност се замислим какво ни трябва, а именно кубитът $\ket{\phi} = \alpha\ket{0} + \beta\ket{1}$ кодиран чрез $ENCODE_1$ и получен след шум като $\ket{\psi}$. Ако се интересуваме само от него можем да приложим операцията на Тофоли\footnote{
\href{https://en.wikipedia.org/wiki/Toffoli_gate}{Гейта на Тофоли} е двойно контролирана $NOT$ операция с матрица: $ CCNOT = \begin{bmatrix}
1 & 0 & 0 & 0 & 0 & 0 & 0 & 0 \\
0 & 1 & 0 & 0 & 0 & 0 & 0 & 0 \\
0 & 0 & 1 & 0 & 0 & 0 & 0 & 0 \\
0 & 0 & 0 & 1 & 0 & 0 & 0 & 0 \\
0 & 0 & 0 & 0 & 1 & 0 & 0 & 0 \\
0 & 0 & 0 & 0 & 0 & 1 & 0 & 0 \\
0 & 0 & 0 & 0 & 0 & 0 & 0 & 1 \\
0 & 0 & 0 & 0 & 0 & 0 & 1 & 0 \\
\end{bmatrix}$
}
и в крайна сметка да елимираме участието на ancilla битове и да обръщаме само $\ket{\psi}$(първия кубит в системата $\ket{\psi}$), ако е необходимо. И тогава нашият алгоритъм схематично ще изглежда така:
\begin{figure}[H]
    \centering
    \begin{math}
    \ket{\psi}
    \begin{cases}
        \begin{array}{c}
        \Qcircuit @C=1em @R=1em {
        \lstick{}
            & \ctrl{1}
            & \ctrl{2}
            & \qw
            & \targ
            & \qw 
            & \rstick{\ket{\phi}}\\
        \lstick{} 
            & \targ
            & \qw
            & \qw 
            & \ctrl{-1}
            & \qw \\
        \lstick{}
            & \qw 
            & \targ
            & \qw
            & \ctrl{-2}
            & \qw 
        }
        \end{array}
    \end{cases}
\end{math}
    \caption{Декодиране чрез операцията на Тофоли}
    \label{fig:sample_decode_toffoli}
\end{figure}
Добавен бонус към това декодиране е, че правим по-малко операции, съответно намалява акумулирания шанс за грешка при прилагане на операциите. \\
Сега за целите на изложението е необходимо да изпишем алгоритъма в явен вид и да пресметнем неговата матрица. Разписваме схемата \figref{fig:sample_decode_toffoli} и получаваме:
\begin{figure}[H]
    \centering
    \begin{math}
    \ket{\psi}
    \begin{cases}
        \begin{array}{c}
        \Qcircuit @C=1em @R=1em {
        \lstick{}
            & \ctrl{1}
            & \qswap
            & \gate{I} 
            & \qswap 
            & \targ
            & \qw
            & \rstick{\ket{\phi}}\\
        \lstick{} 
            & \targ
            & \qswap \qwx
            & \ctrl{1}
            & \qswap \qwx
            & \ctrl{-1}
            & \qw \\
        \lstick{}
            & \gate{I} 
            & \gate{I} 
            & \targ
            & \gate{I} 
            & \ctrl{-2}
            & \qw 
        }
        \end{array}
    \end{cases}
\end{math}
    \caption{Декодиране чрез операцията на Тофоли(разписна версия)}
    \label{fig:sample_decode_toffoli_expanded}
\end{figure}
Въпреки, че опрацията на Тофоли е дефинирана при:
\begin{figure}[H]
    \centering
    \begin{math}
    \begin{array}{c}
    \Qcircuit @C=1em @R=1em {
        & \ctrl{1} & \qw\\
        & \ctrl{1} & \qw\\
        & \targ & \qw
    }
    \end{array}
\end{math}
    \caption{Стандартната операция на Тофоли}
    \label{fig:standard_op_Toffoli}
\end{figure}
Но много лесно можем да рапределяме "контролите"\footnote{
Контролите на една операция са изобразени като точки на схемите и служат за обозначаване на това, че дадена операция се контролира от кубитът в линията, където е точката.
}, където поискаме чрез умножение по $SWAP$ матрицата(отляво или отдясно) дадена в \eqref{eq:cnot_and_swap_matrices}. Често това не се отразява експлицитно, защото се счита, че винаги можем да преномерираме кубитовете в края и да получим желания резултат. С цел експицитност на изложението ще извършвам и тези операции, тъй като това допринася за яснота на цялостното разбиране. В случая резултата е очевитен:
\begin{figure}[H]
    \centering
    \begin{math}
    \begin{array}{c c c}
    \Qcircuit @C=1em @R=1em {
        & \targ & \qw\\
        & \ctrl{-1} & \qw\\
        & \ctrl{-1} & \qw
    }
    & \iff &
    \Qcircuit @C=1em @R=1em {
        & \qswap
        & \qw
        & \ctrl{1} 
        & \qw
        & \qswap
        & \qw\\
        & \qswap \qwx
        & \qswap
        & \ctrl{1}
        & \qswap
        & \qswap \qwx
        & \qw\\
        & \qw
        & \qswap \qwx
        & \targ 
        & \qswap \qwx
        & \qw
        & \qw
    }
    \end{array}
\end{math}
    \caption{Разписана модифицираната операция на Тофоли от \figref{fig:sample_decode_toffoli}}
    \label{fig:toffolli_decode_expanded}
\end{figure}
\begin{equation}
    NOTCC\footnotemark = \begin{bmatrix}
    1 & 0 & 0 & 0 & 0 & 0 & 0 & 0 \\
    0 & 1 & 0 & 0 & 0 & 0 & 0 & 0 \\
    0 & 0 & 1 & 0 & 0 & 0 & 0 & 0 \\
    0 & 0 & 0 & 0 & 0 & 0 & 0 & 1 \\
    0 & 0 & 0 & 0 & 1 & 0 & 0 & 0 \\
    0 & 0 & 0 & 0 & 0 & 1 & 0 & 0 \\
    0 & 0 & 0 & 0 & 0 & 0 & 1 & 0 \\
    0 & 0 & 0 & 1 & 0 & 0 & 0 & 0 
    \end{bmatrix}
    \label{eq:notcc_matrix}
\end{equation}
\footnotetext{
Прави ни впечатление, че всъщност матрицата на $NOTCC$ не е много по-различна от тази на $CCNOT$, дори можем да забележим, че е просто "транспонирана по обратния диагонал"
}
След като за стандартната трансформация използвахме $CCNOT$ и нейните контроли са към "горните"\ кубитове, то логично можем да заключим, че можем да смятаме наименованието като всяка паралелна операция се поставя след "горната". Щом $CCNOT$ е Controlled-Controlled-NOT, то най-логичното наименованите на получената матрица е $NOTCC$ - NOT-Controlled-Controlled.\\
След като имаме \eqref{eq:cnot_and_swap_matrices} и \eqref{eq:notcc_matrix} можем да пресметнем матрицата на $DECODE_1$ операцията от \figref{fig:sample_decode_toffoli_expanded}
\begin{equation}
    DECODE_1 = NOTCC . (SWAP \otimes I) . (I \otimes CNOT) . (SWAP \otimes I) . (CNOT \otimes I)
    \label{eq:decode_matrix_formula}
\end{equation}
Сега извършваме операциите в \eqref{eq:decode_matrix_formula} и получаваме:
\begin{equation}
    DECODE_1 = \begin{bmatrix}
    1 & 0 & 0 & 0 & 0 & 0 & 0 & 0\\
    0 & 1 & 0 & 0 & 0 & 0 & 0 & 0\\
    0 & 0 & 1 & 0 & 0 & 0 & 0 & 0\\
    0 & 0 & 0 & 0 & 1 & 0 & 0 & 0\\
    0 & 0 & 0 & 0 & 0 & 0 & 0 & 1\\
    0 & 0 & 0 & 0 & 0 & 0 & 1 & 0\\
    0 & 0 & 0 & 0 & 0 & 1 & 0 & 0\\
    0 & 0 & 0 & 1 & 0 & 0 & 0 & 0
    \end{bmatrix}
\end{equation}
\paragraph{Примерни изчисления с декодиращата матрица(проверка)}
Сега, за да се уверим, че наистина това е матрицата на декодирането ще я приложим към резултатите от: \eqref{eq:simple_encode_000}, \eqref{eq:simple_encode_100}, \eqref{eq:simple_encode_+00}, \eqref{eq:simple_encode_-00}\\
$DECODE_1$ приложено на резултата от \eqref{eq:simple_encode_000}
\begin{equation}
    DECODE_1 . \ket{000} = \ket{000}
\end{equation}
\begin{equation}
    DECODE_1 . \ket{111} = \ket{100}
\end{equation}
\begin{equation}
    DECODE_1 . \ket{+00} = \ket{+00}
\end{equation}
\begin{equation}
    DECODE_1 . \ket{-00} = \ket{-00}
\end{equation}
Уверихме се, че декодира правилно изчислените състояния, какво ще получим обаче, ако получим системата $\ket{\psi} = \ket{011}$, бихме очаквали да има шум в първия кубит от системата(именно този, който ни интересува), затова прилагаме $DECODE_1$ и получаваме
\begin{equation}
    DECODE_1 . \ket{011} = \ket{111}
\end{equation}
И както можем да забележим първия кубит се обръща и получаваме това, което сме очаквали.\\
Нека разгледаме по-особен случай $\ket{\psi} = \alpha\ket{010} + \beta\ket{101}$
\begin{equation}
    DECODE_1 . \ket{\psi} = \alpha\ket{010} + \beta\ket{110}
\end{equation}
Както е видно системата не прие състоянието, което бихме очаквали при пълно декориране, в случая получихме частично декодиране(само върху първия кубит в системата).

\section{Недостатъци на метода}
\paragraph{Phase-flip errors}
Гореизложения метод(вж. \figref{fig:sample_coding} и \figref{fig:sample_decode_toffoli}) могат да засичат и коригират грешки от т. нар. тип "bit-flip errors", но могат да настъпят и "phase-flip errors"\footref{ftnt:phase_flip}. Затова е необходимо разглеждане на втори алгоритъм именно за този вид грешки. Ако разгледаме какво всъщност представляват "phase-flip" грешките:
\begin{equation}
    \alpha\ket{0} + \beta\ket{1} \xleftrightarrow{\text{Phase flip}} \alpha\ket{0} - \beta\ket{1}
\end{equation}
В частност:
\begin{equation}
    \ket{+} = \frac{1}{\sqrt{2}}\ket{0} + \frac{1}{\sqrt{2}}\ket{1} \xleftrightarrow{\text{Phase flip}} \frac{1}{\sqrt{2}}\ket{0} - \frac{1}{\sqrt{2}}\ket{1} = \ket{-}
    \label{eq:phase_flip_diag_basis}
\end{equation}
От \eqref{eq:phase_flip_diag_basis} можем да забележим, че "phase-flip" грешката в правоъгълния базис може да се разглежда като "bit-flip" грешка в диагоналния. Следователно нашият алгоритъм трябва да "премине" от единия базис в другия. Можем да използваме квантовата Фурие трансформация\footnote{
\href{https://en.wikipedia.org/wiki/Quantum_Fourier_transform}{Квантовата Фурие трансформация} ни позволява да преминаваме от единия базис в другия като ни предоставя формула за матрицата на трансформацията
}, но за целите на кодиране и декодиране можем да използваме опростена схема с Хадамар трансформацията.
\begin{figure}[H]
    \centering
    \begin{math}
    \Qcircuit @C=1em @R=1em {
        \lstick{\ket{\phi}}
        & \ctrl{1} 
        & \ctrl{2}
        & \gate{H}
        & \multigate{2}{NOISE}
        & \gate{H}
        & \ctrl{1} 
        & \ctrl{2}
        & \targ
        & \qw
        & \rstick{\ket{\phi}}\\
        \lstick{\ket{0}}
        & \targ 
        & \qw
        & \gate{H}
        & \ghost{NOISE}
        & \gate{H}
        & \targ 
        & \qw
        & \ctrl{-1}
        & \qw\\
        \lstick{\ket{0}}
        & \qw
        & \targ
        & \gate{H}
        & \ghost{NOISE}
        & \gate{H}
        & \qw
        & \targ
        & \ctrl{-1}
        & \qw
    }
    \end{math}
    \caption{Схема за засичане и коригиране на phase-flip грешки}
    \label{fig:phase_flip_encode_decode}
\end{figure}
Отново сме се възползвали от идеята, че се интересуваме само от първия кубит на системата, а не от ancilla битовете. Матриците на $ENCODE_{phase}$ и $DECODE_{phase}$ могат тривиално да бъдат изписани от горната схема, чрез пресмятания подобни на \eqref{eq:encode_matrix_formula} и \eqref{eq:decode_matrix_formula}
\section{Кодировка на Шор} \cite{shor-code}
\paragraph{История}
Кодировката разработена от \href{https://en.wikipedia.org/wiki/Peter_Shor}{Питър Шор}. Чрез нея могат да се коригират произволни линейни комбинации грешки върху \emph{единствен} кубит. Подхода е подобен на съвременни кодировки, а именно подхода на разделяне между логически и физически еденици. При кодировката на Шор един логически кубит(с който правим изчисления) се представя чрез 9 физически кубита. Той съчетава двете гореизложени схеми. Неговата схема изглежда така:
\begin{figure}[H]
    \centering
    \begin{math}
    \Qcircuit @C=1em @R=1em {
    \lstick{\ket{\phi}}
        & \ctrl{3}
        & \ctrl{6}
        & \gate{H}
        & \ctrl{1}
        & \ctrl{2}
        & \multigate{8}{NOISE}
        & \ctrl{2}
        & \ctrl{1}
        & \targ
        & \gate{H}
        & \ctrl{6}
        & \ctrl{3}
        & \targ
        & \qw
        & \rstick{\ket{\phi}}\\
    \lstick{\ket{0}}
        & \qw
        & \qw
        & \qw
        & \targ
        & \qw
        & \ghost{NOISE}
        & \qw
        & \targ
        & \ctrl{-1}
        & \qw
        & \qw
        & \qw
        & \qw
        & \qw\\
    \lstick{\ket{0}}
        & \qw
        & \qw
        & \qw
        & \qw
        & \targ
        & \ghost{NOISE}
        & \targ
        & \qw
        & \ctrl{-1}
        & \qw
        & \qw
        & \qw
        & \qw
        & \qw\\
    \lstick{\ket{0}}
        & \targ
        & \qw
        & \gate{H}
        & \ctrl{1}
        & \ctrl{2}
        & \ghost{NOISE}
        & \ctrl{2}
        & \ctrl{1}
        & \targ
        & \gate{H}
        & \qw
        & \targ
        & \ctrl{-3}
        & \qw\\
    \lstick{\ket{0}}
        & \qw
        & \qw
        & \qw
        & \targ
        & \qw
        & \ghost{NOISE}
        & \qw
        & \targ
        & \ctrl{-1}
        & \qw
        & \qw
        & \qw
        & \qw
        & \qw\\
    \lstick{\ket{0}}
        & \qw
        & \qw
        & \qw
        & \qw
        & \targ
        & \ghost{NOISE}
        & \targ
        & \qw
        & \ctrl{-1}
        & \qw
        & \qw
        & \qw
        & \qw
        & \qw\\
    \lstick{\ket{0}}
        & \qw
        & \targ
        & \gate{H}
        & \ctrl{1}
        & \ctrl{2}
        & \ghost{NOISE}
        & \ctrl{2}
        & \ctrl{1}
        & \targ
        & \gate{H}
        & \targ
        & \qw
        & \ctrl{-3}
        & \qw\\
    \lstick{\ket{0}}
        & \qw
        & \qw
        & \qw
        & \targ
        & \qw
        & \ghost{NOISE}
        & \qw
        & \targ
        & \ctrl{-1}
        & \qw
        & \qw
        & \qw
        & \qw
        & \qw\\
    \lstick{\ket{0}}
        & \qw
        & \qw
        & \qw
        & \qw
        & \targ
        & \ghost{NOISE}
        & \targ
        & \qw
        & \ctrl{-1}
        & \qw
        & \qw
        & \qw
        & \qw
        & \qw
    }
    \end{math}
    \caption{Кодировка на Шор}
    \label{fig:shors_encoding}
\end{figure}
Интересното, което може да се отбележи за тази кодировка е, че кубитовете на местата 0,3,6 се използват за засичане и коригиране на "phase-flip" грешки, а кубитовете по групи - (0,1,2), (3,4,5), (6,7,8) аналогично за "bit-flip" грешки. В общия случай кодировката на Шор може да се използва за грешки от вида на \figref{fig:noise_gate}:
\begin{equation}
    NOISE = c_1 . I + c_2 . \sigma_x  + c_3 . \sigma_y  + c_4 . \sigma_z
\end{equation}
Където $c_1, c_2, c_3 \in \mathbb{C}$ и
\begin{equation}
\begin{multlined}
    \sigma_x = \begin{bmatrix}
    0 & 1 \\
    1 & 0
    \end{bmatrix}\\
    \sigma_y = \begin{bmatrix}
    0 & -i \\
    i & 0
    \end{bmatrix}\\
    \sigma_z = \begin{bmatrix}
    1 & 0 \\
    0 & -1
    \end{bmatrix}\\
\end{multlined}
\label{eq:pauli_matrices}
\end{equation}
Съответно разглеждани: $\sigma_x$ - bit-flip, $\sigma_y$ - phase-flip, $\sigma_z$ - и двете
\section{Заключение}
Настоящата разработка разгледа кодировки на един логически кубит и подходи за откриване и поправяне на единични грешки в системите на същия този логически кубит. Необходимо е да се отбележи, че тази област е обект на продължителни разработки, съответно изложените методи могат да търпят промени с течение на времето и да бъдат усъвършенствани.\\
Изчисленията и проверките на операциите бяха проведени на симулация на квантов компютър \cite{Simulation-Q}. Също така матричните изчисления бяха улеснени от математическия модул на същата тази симулация.
\newpage
\printbibliography[title=Цитирани ресурси]
\end{document}
